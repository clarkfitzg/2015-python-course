% A simple template for LaTeX documents

\documentclass[11pt]{article}

% Begin paragraphs with new line
\usepackage{parskip}  

% Change margin size
\usepackage[margin=1in]{geometry}   

% Graphics Example:  (PDF's make for good plots)
% \includegraphics{filename}  % No file extension or quotes needed
\usepackage{graphicx}               

% Allows hyperlinks
\usepackage{hyperref}

% Blocks of code
\usepackage{listings}
\lstset{basicstyle=\ttfamily, title=\lstname}


\begin{document}

\section*{Makefiles}

Originally designed for compiling large software projects, Makefiles can automate the dependencies between data analysis steps, enabling \textbf{reproducible research}.\footnote{For more on reproducible research, check out the chapter on Open Source Scientific Practice by K. Jarrod Millman and Fernando Perez available at \url{https://osf.io/h9gsd/}.}
Taking a little time early on to precisely record and understand your steps in a Makefile will pay big dividends when it comes time to reproduce that analysis.

To create this pdf, navigate to the directory containing the Makefile and type {\tt make}. Note that you will need \LaTeX, R, Graphviz, and Python installed for this example. The Makefile contains a list of dependencies and commands that act as a recipe for building the final product, in this case a pdf. Later, if you change any step in the analysis you can issue the {\tt make} command again and the make program will run the appropriate commands to refresh the results.

Makefiles consist of rules that follow this pattern:

\begin{lstlisting}
target: dependencies
[tab] system command
\end{lstlisting}

Note that it \textbf{must} have a tab character, spaces will not work. You may need to change your text editor settings. This is one of the only times that tab characters are a good idea when programming. The Makefile below generates this pdf.

\vspace{0.3in}

\lstinputlisting{Makefile}

\vspace{0.3in}

Below is a visualization of the Makefile. A Python script parsed the Makefile and created a file in the dot language. The dot language uses syntax like {\tt a -> b} to draw a line from a to b. It's as easy at it sounds. The magic happens when the dot file is dynamically generated from some underlying data, as in this case. Graphviz\footnote{\url{http://www.graphviz.org/}} takes the dot language as input and renders the structure in an aesthetically pleasing way.

\centerline{\includegraphics{make_graph.pdf}}

The first line describes our final target, {\tt paper.pdf}. It says that {\tt paper.pdf} depends on the \TeX\ source {\tt paper.tex}, {\tt figure.pdf}, {\tt make\_graph.pdf}, {\tt plot.R},  and {\tt Makefile}. If any of these files change then the corresponding command {\tt pdflatex paper.tex} will run, which makes a new {\tt paper.pdf}. It's a little strange that it directly depends on all these source files but the reason is that \LaTeX\ uses the {\tt listings} package to actually pull in the source code of {\tt Makefile} and {\tt plot.R} and render it in the pdf!

The line with {\tt figure.pdf} as the target has dependencies {\tt data.csv} and {\tt plot.R}. If the data or the R script changes then {\tt make} needs to create a new figure.

{\tt data.csv} is a target with dependency {\tt get\_data.py}. If the Python script that produces the data changes and we run {\tt make} again then the command {\tt python get\_data.py} will run, updating {\tt data.csv}. This will cause {\tt figure.pdf} to be plotted again, which will in turn cause the final pdf to be recreated.

The final target is {\tt clean}. Issuing the command {\tt make clean} runs the line beginning with {\tt rm}, removing the byproducts of R and \LaTeX.

{\tt make} is great because it's lazy. Good programmers always look out for a chance to be lazy. That means that if you only change {\tt paper.tex}, then {\tt make} won't plot the figure or do anything with the python script.This becomes important if any of the steps take a long time.

\subsubsection*{R graphics}

Below is an example of including graphics created by R. The source file below generates the corresponding plot.

\lstinputlisting{plot.R}

R saved this plot as a PDF. PDF's are vector graphics, which means that they'll look good no matter how much you zoom. Use them for professional results. Mathematical expressions are always beautiful in \LaTeX. 
\[
    f(x) = x^2 + 5x + \pi
\]

\centerline{\includegraphics{figure.pdf}}

\end{document}
